% Created 2012-04-24 Tue 02:10
\documentclass[11pt]{article}
\usepackage[utf8]{inputenc}
\usepackage[T1]{fontenc}
\usepackage{graphicx}
\usepackage{longtable}
\usepackage{float}
\usepackage{wrapfig}
\usepackage{soul}
\usepackage{amssymb}
\usepackage{hyperref}


\title{CIS192 Project Overview}
%\author{Nicholas J Watson}
\date{2012-04-24}

\begin{document}

\maketitle



\section{Source}
\label{sec-1}

  Code is available at \href{https://github.com/zwass/Sneaky}{https://github.com/zwass/Sneaky}

\section{Group}
\label{sec-2}

\begin{itemize}
\item Nick Watson (nwatson)
\item Zachary Wasserman (zwass)
\end{itemize}
\section{Description}
\label{sec-3}

  We built a system to encode hidden messages into images. As
  discussed in CIS551 (Computer and Network Security), one can
  slightly modify the pixels of an image in order to encode data,
  without making changes visible to a casual observer. This process is
  called ``watermarking'', and is a type of steganography, which is the
  practice of concealing a message to hide its existance. This is
  different from standard cryptography, which relies on mathematically
  proven methods to transform plaintext to ciphertext, which is plainly
  visible but hard to decrypt without some secret knowledge (e.g. a key).

  Our scheme allows for the encoding of a relatively large amount of
  data onto a source image. For example, we were able to encode the
  entire script of Hamlet onto a portrait of William
  Shakespeare. Using a parity-bit encoding, we can allow decoding
  without even requiring a reference image.

  Encoding and decoding are quite fast. First we take the input
  data and convert it to a bitstring. We then traverse the image,
  slightly modifying the red value of each pixel to encode a single bit.
  On decoding,
  we can examine the parity of the red-values, and from it recover the
  message. Using a medium-powered laptop computer, we are able to
  perform these actions on the order of a few seconds.

  We attempted an optimization by using bitwise and arithmetic operations
  instead of string operations, but this did not provide a significant
  speed-up. This is most likely due to the fact that Python is too
  far-removed from hardware to gain the same time benefit from these operations
  as a lower-level language like C would.

  Our software allows for a fun and somewhat covert transmission of
  data. Obviously, there is no actual encryption, but by appearing to
  be normal images, the encoded images that we post are likely to
  afford ``security through obscurity.''

  Lists are the primary data structure harnessed by our software. We
  use the pixel matrix provided by the Python Imaging Library (PIL), and
  single dimensional lists for our encoded data and
  bitstrings. Network communication is mostly assisted by the TwitPic
  and Tweepy API wrappers for TwitPic and Twitter, which provide
  abstractions over the OAuth and REST based APIs provided by the
  services.

  The biggest challenge in creating this software was working with the
  PIL to perform the actual encoding and decoding of messages onto the
  images. Any single error in our encoding resulted in an impossible
  to decode message. Ultimately we found that starting small and
  working up to the more complicated encoding scheme, and larger images
  and messages was helpful in correctly implementing the project.

  Working on this project afforded us a broad range of experience of
  developing Python applications. We learned how to manipulate images
  and files, as well as interacting with third party APIs using the
  assistance of publicly available Python libraries.

  Had we had more time, we would have liked to build a GUI for working
  with the images and messages. This seemed like it could have been an
  entire project in itself, so we instead chose to focus on
  functionality.

  An additional feature that might be fun to have is the ability to do
  both cryptography and steganography simultaneously. Before encoding the
  message into the image, it could be encrypted using RSA or AES. This
  would make it virtually impossible to determine whether an image actually
  contained a message, since running the decoding algorithm would appear
  to output gibberish if the incorrect key were given.

\section{Features}
\label{sec-4}

\begin{itemize}
\item Encoding and decoding of hidden data in watermarked images
\item Only one image is required to decode a message. The encoding
    scheme uses a parity bit to determine the hidden data.
\item Integration with Twitter

\begin{itemize}
\item Automatically encode and post messages to Twitter
\item Decode messages from images in a Twitter stream
\end{itemize}

\end{itemize}
\section{Libraries}
\label{sec-5}

\begin{itemize}
\item Python Imaging Library (\href{http://www.pythonware.com/products/pil/}{http://www.pythonware.com/products/pil/})
\item Python Requests
    (\href{http://docs.python-requests.org/en/v0.10.7/index.html}{http://docs.python-requests.org/en/v0.10.7/index.html})
\item Python TwitPic (\href{https://github.com/macmichael01/python-twitpic}{https://github.com/macmichael01/python-twitpic})
\item Tweepy (\href{http://code.google.com/p/tweepy/}{http://code.google.com/p/tweepy/})
\end{itemize}
\section{Responsibilities}
\label{sec-6}

\subsection{Nick}
\label{sec-6.1}

\begin{itemize}
\item Parity bit encoding/decoding
\item Twitter integration
\end{itemize}
\subsection{Zach}
\label{sec-6.2}

\begin{itemize}
\item Basic encoding/decoding
\item Twitter integration
\end{itemize}

\end{document}